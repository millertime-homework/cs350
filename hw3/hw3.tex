% Homework 3 - CS350
% Russell Miller Winter 2011

\documentclass{article}
\usepackage{anysize}
\usepackage{fullpage}
\usepackage{cancel}

\marginsize{2cm}{2cm}{2cm}{2cm}

\title{CS350 Homework 3}
\author{Russell Miller}
\date{\today}

\begin{document}

\maketitle

\section*{22.1-1}
The out-degree of a vertex is just its adjacency list, so in $\Theta(V)$
time you can get the out-degree of every vertex.\\
Finding the in-degree for every vertex would require looking for that
vertex in every vertex's adjacency list. So the entire adjacency list 
must be visited, which would be $\Theta(V*E)$, where $V$ is the total
number of vertices, and $E$ is the total number of edges.\\

\section*{22.1-3}
\begin{verbatim}
Adjacency List
[(1,[2,3]),
 (2,[4,5]),
 (3,[6,7])]

Adjacency Matrix
\end{verbatim}
\begin{tabular}{|c|c c c c c c c|}
\hline
 &1&2&3&4&5&6&7\\
\hline
1&0&1&1&0&0&0&0\\
2&0&0&0&1&1&0&0\\
3&0&0&0&0&0&1&1\\
4&0&0&0&0&0&0&0\\
5&0&0&0&0&0&0&0\\
6&0&0&0&0&0&0&0\\
7&0&0&0&0&0&0&0\\
\hline
\end{tabular}

\section*{22.1-3}


\end{document}
