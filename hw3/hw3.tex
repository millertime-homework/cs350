% Homework 3 - CS350
% Russell Miller Winter 2011

\documentclass{article}
\usepackage{anysize}
\usepackage{fullpage}
\usepackage{cancel}

\marginsize{2cm}{2cm}{2cm}{2cm}

\title{CS350 Homework 3}
\author{Russell Miller}
\date{\today}

\begin{document}

\maketitle

\section*{22.1-1}
The out-degree of a vertex is just its adjacency list, so in $\Theta(V)$
time you can get the out-degree of every vertex.\\
Finding the in-degree for every vertex would require looking for that
vertex in every vertex's adjacency list. So the entire adjacency list 
must be visited, which would be $\Theta(V*E)$, where $V$ is the total
number of vertices, and $E$ is the total number of edges.\\

\section*{22.1-3}
\begin{verbatim}
Adjacency List
[(1,[2,3]),
 (2,[4,5]),
 (3,[6,7])]

Adjacency Matrix
\end{verbatim}
\begin{tabular}{|c|c c c c c c c|}
\hline
 &1&2&3&4&5&6&7\\
\hline
1&0&1&1&0&0&0&0\\
2&0&0&0&1&1&0&0\\
3&0&0&0&0&0&1&1\\
4&0&0&0&0&0&0&0\\
5&0&0&0&0&0&0&0\\
6&0&0&0&0&0&0&0\\
7&0&0&0&0&0&0&0\\
\hline
\end{tabular}

\section*{22.1-3}
\begin{verbatim}
TRANSPOSE(ajd_list A)
1  new adj_list B
2  for each v in A
3    for each (v,x) in v
4      B.add(x,v)

TRANSPOSE(adj_matrix A)
1  new ajd_matrix B[A.width][A.height]
2  for i in A.width
3    for j in A.height
4      B[j][i] = A[i][j]
\end{verbatim}
The running time of the list version of the transpose is $O(|V|^2)$, where $V$ is the number of vertices.
In the best case, with a completely disconnected graph, it would only take $|V|$.\\
\\
The running time of the matrix version is $\Theta(|V|^2)$, because it copies the entire matrix which
is width and height $V$.\\

\section*{22.1-6}
Finding a universal sink has 3 parts: identifying the correct vertex, verifying that its in-degree is
$|V|-1$, and verifying that its out-degree is $0$.\\
\\
In order to find a vertex with zero out-degree, the worst-case would be $|V|^2$, so immediately we know this
is not where to start. We also know that determining a vertex's in-degree is $|V|^2$.\\
\\
So as a shortcut, we'll start from an arbitrary row in the matrix, and when a $1$ is found in column $j$,
move to row $i$, where $i = j$. Then find the first $1$, repeating this process until a row $i$ is found with all
zeros. This could potentially be the universal sink. The last step is to verify that in column $j$, where $j = i$,
there are all $1$s.\\
While jumping from row to row, there were a few traversals within each row, for less than $V$ rows. There was also
the traversal of the sink row, which is length $V$. And lastly, there was the traversal of the $V$-length column.
This is $3V$, which is $O(V)$.\\
\\
This is clever, but it's not actually complete. This is because there are cases when this would be much worse than
$O(V)$. Here is the matrix to show a counterexample. Imagine a graph where all vertices are connected to each other,
AND the sink, and the algorithm traverses every edge before reaching the sink. It would take $O(V^2)$.\\
\begin{tabular}{|c|c c c c c c c|}
\hline
  & 1 & 2 & 3 & 4 & 5 & 6 & 7\\
\hline
1 & 0 & 1 & 0 & 0 & 0 & 0 & 1\\
2 & 0 & 0 & 1 & 0 & 0 & 0 & 1\\
3 & 0 & 0 & 0 & 1 & 0 & 0 & 1\\
4 & 0 & 0 & 0 & 0 & 1 & 0 & 1\\
5 & 0 & 0 & 0 & 0 & 0 & 1 & 1\\
6 & 0 & 0 & 0 & 0 & 0 & 0 & 1\\
7 & 0 & 0 & 0 & 0 & 0 & 0 & 0\\
\hline
\end{tabular}

\section*{22.2-8}

\end{document}
