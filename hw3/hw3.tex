% Homework 3 - CS350
% Russell Miller Winter 2011

\documentclass{article}
\usepackage{anysize}
\usepackage{fullpage}
\usepackage{cancel}

\marginsize{2cm}{2cm}{2cm}{2cm}

\title{CS350 Homework 3}
\author{Russell Miller}
\date{\today}

\begin{document}

\maketitle

\section*{22.1-1}
The out-degree of a vertex is just its adjacency list, so in $\Theta(V)$
time you can get the out-degree of every vertex.\\
Finding the in-degree for every vertex would require looking for that
vertex in every vertex's adjacency list. So the entire adjacency list 
must be visited, which would be $\Theta(V*E)$, where $V$ is the total
number of vertices, and $E$ is the total number of edges.\\

\section*{22.1-3}
\begin{verbatim}
Adjacency List
[(1,[2,3]),
 (2,[4,5]),
 (3,[6,7])]

Adjacency Matrix
\end{verbatim}
\begin{tabular}{|c|c c c c c c c|}
\hline
 &1&2&3&4&5&6&7\\
\hline
1&0&1&1&0&0&0&0\\
2&0&0&0&1&1&0&0\\
3&0&0&0&0&0&1&1\\
4&0&0&0&0&0&0&0\\
5&0&0&0&0&0&0&0\\
6&0&0&0&0&0&0&0\\
7&0&0&0&0&0&0&0\\
\hline
\end{tabular}

\section*{22.1-3}
\begin{verbatim}
TRANSPOSE(ajd_list A)
1  new adj_list B
2  for each v in A
3    for each (v,x) in v
4      B.add(x,v)

TRANSPOSE(adj_matrix A)
1  new ajd_matrix B[A.width][A.height]
2  for i in A.width
3    for j in A.height
4      B[j][i] = A[i][j]
\end{verbatim}
The running time of the list version of the transpose is $O(|V|^2)$, where $V$ is the number of vertices.
In the best case, with a completely disconnected graph, it would only take $|V|$.\\
\\
The running time of the matrix version is $\Theta(|V|^2)$, because it copies the entire matrix which
is width and height $V$.\\

\section*{22.1-6}
Finding a universal sink has 3 parts: identifying the correct vertex, verifying that its in-degree is
$|V|-1$, and verifying that its out-degree is $0$.\\
\\
In order to find a vertex with zero out-degree, the worst-case would be $|V|^2$, so immediately we know this
is not where to start. We also know that determining a vertex's in-degree is $|V|^2$.\\
\\
So as a shortcut, we'll start from an arbitrary row in the matrix, and when a $1$ is found in column $j$,
move to row $i$, where $i = j$. Then find the first $1$, repeating this process until a row $i$ is found with all
zeros. This could potentially be the universal sink. The last step is to verify that in column $j$, where $j = i$,
there are all $1$s.\\
While jumping from row to row, there were a few traversals within each row, for less than $V$ rows. There was also
the traversal of the sink row, which is length $V$. And lastly, there was the traversal of the $V$-length column.
This is $3V$, which is $O(V)$.\\
\\
This is clever, but it's not actually complete. This is because there are cases when this would be much worse than
$O(V)$. Here is the matrix to show a counterexample. Imagine a graph where all vertices are connected to each other,
AND the sink, and the algorithm traverses every edge before reaching the sink. It would take $O(V^2)$.\\
\begin{tabular}{|c|c c c c c c c|}
\hline
  & 1 & 2 & 3 & 4 & 5 & 6 & 7\\
\hline
1 & 0 & 1 & 0 & 0 & 0 & 0 & 1\\
2 & 0 & 0 & 1 & 0 & 0 & 0 & 1\\
3 & 0 & 0 & 0 & 1 & 0 & 0 & 1\\
4 & 0 & 0 & 0 & 0 & 1 & 0 & 1\\
5 & 0 & 0 & 0 & 0 & 0 & 1 & 1\\
6 & 0 & 0 & 0 & 0 & 0 & 0 & 1\\
7 & 0 & 0 & 0 & 0 & 0 & 0 & 0\\
\hline
\end{tabular}

\section*{22.2-8}
\begin{verbatim}
MAZESOLVE(G)
1  new array visits[ G.edges.length ]
2  new array status[ G.vertices.length]
3  for each v in G.vertices
4    status[v] = unvisited
5  endfor
6  new queue Q
7  Q.enqueue(G.vertices[0])
8  status[G.vertices[0]] = discovered
9  while Q.notempty
10   u = Q.dequeue
11   for each v in u.adj_list
12     if status[v] == unvisited
13       ++visits[(u,v)]
14       status[v] = discovered
15       Q.enqueue(v)
16     endif
17   endfor
18   status[u] = visited
19   for each v in u.adj_list
20     ++visits[(u,v)]
21   endfor
22 endwhile
\end{verbatim}
This algorithm should be setting each value of the 'visits' array to 2.
That means each edge has been visited twice. The way this can solve a maze
is by traversing the graph, you can find your way back out by following the
vertices you've marked twice. Those are the ones that are visited.\\

\section*{22.3-8}
Using adjacency-list representation, here is an example graph:
\begin{verbatim}
{ A : (B,C),
  B : (),
  C : (B) }
\end{verbatim}
The depth-first search would produce a time table such as this:\\
\begin{tabular}{|c|c c c|}
\hline
  & A & B & C\\
\hline
d & 1 & 2 & 4\\
\hline
v & 6 & 3 & 5\\
\hline
\end{tabular}
\\This includes a path $C \rightarrow B$, where $d[B] < v[C]$.\\

\section*{22.3-11}
I chose to write this algorithm in Python, here is the code:
\begin{verbatim}
#!/usr/bin/python

import sys

def main():
    test_graph = Graph()
    components,n = test_graph.depth_first_search()
    i = 1
    for c in components:
        sys.stdout.write("cc[%d] = " % i)
        i += 1
        print(c)
    print("%d total components" % n)

class Graph:
    """This is a single-use graph class
    It's going to traverse a disconnected graph
    and discover the graph's connected components"""
    def __init__(self):
        self.components = []
        # this is a python dict -- key,value pair (separated by ':') 
        # of vertex and adjacency list
        self.graph = { 'A': ['B',],
                       'B': ['A','C'],
                       'C': ['B',],
                       'D': ['E',],
                       'E': ['D',],
                       'F': ['G',],
                       'G': ['F','H'],
                       'H': ['G',] }
        # this is the 'color' label for each vertex: 
        #    unvisited, discovered, or visited
        self.status = {}
        for v in self.graph.keys():   # keys is the vertices from the graph dict
            self.status[v] = 'unvisited'
        
    def depth_first_search(self):
        """this is the algorithm from the slides"""
        for v in self.graph.keys():
            if self.status[v] == 'unvisited':
                self.components.append([])
                self.visit(v)
        return self.components,len(self.components)

    def visit(self, v):
        self.status[v] = 'discovered'
        if v not in self.components[len(self.components)-1]:
            self.components[len(self.components)-1].append(v)
        for u in self.graph[v]:
            if u not in self.components[len(self.components)-1]:
                self.components[len(self.components)-1].append(u)
            if self.status[u] == 'unvisited':
                self.visit(u)
            self.status[v] = 'visited'

if __name__=="__main__":
    main()
\end{verbatim}
Here is the output from running the program:
\begin{verbatim}
cc[1] = ['A', 'B', 'C']
cc[2] = ['E', 'D']
cc[3] = ['G', 'F', 'H']
3 total components
\end{verbatim}

\section*{22.4-3}
It's not possible.\\
Actually, I'm just giving up.\\
It turns out, this took $O(1)$ time.\\

\end{document}
